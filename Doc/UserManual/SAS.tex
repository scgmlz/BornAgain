%%%%%%%%%%%%%%%%%%%%%%%%%%%%%%%%%%%%%%%%%%%%%%%%%%%%%%%%%%%%%%%%%%%%%%%%%%%%%%%%
%%
%%   BornAgain User Manual
%%
%%   homepage:   http://www.bornagainproject.org
%%
%%   copyright:  Forschungszentrum Jülich GmbH 2015
%%
%%   license:    Creative Commons CC-BY-SA
%%
%%   authors:    Scientific Computing Group at MLZ Garching
%%               C. Durniak, M. Ganeva, G. Pospelov, W. Van Herck, J. Wuttke
%%
%%%%%%%%%%%%%%%%%%%%%%%%%%%%%%%%%%%%%%%%%%%%%%%%%%%%%%%%%%%%%%%%%%%%%%%%%%%%%%%%

\part{Physics}\label{PPHYS}

\chapter{Foundations of small-angle scattering}  \label{SSas}
\chaptermark{Small-angle scattering}

\index{Small-angle scattering|(}%

This chapter introduces the basic theory of small-angle scattering (SAS).
We specifically consider scalar neutron propagation,
adjourning the notationally more involved
vectorial theory of X-rays and polarized neutrons
to a later edition. % TODO REVERT TEMPORARY % to the later \cref{SPol}.
Our exposition is self-contained,
except for the initial passage from the microscopic
to the macroscopic Schrödinger equation,
which we outline only briefly (\Cref{Swave}).
The standard description of scattering in first order Born approximation
is introduced in a way that is suitable subsequent modification
into the distorted wave Born approximation
needed for grazing-incidence small-angle scattering (\Cref{SBornApprox}).

Finally, we discuss how the physical scattering law
(the double differential cross section as function of the scattering vector~$\q$)
relates to the experimental detector image (Sec.\ref{SdetImg}).


%%%%%%%%%%%%%%%%%%%%%%%%%%%%%%%%%%%%%%%%%%%%%%%%%%%%%%%%%%%%%%%%%%%%%%%%%%%%%%%%
\section{Coherent neutron propagation}\label{Swave}
%%%%%%%%%%%%%%%%%%%%%%%%%%%%%%%%%%%%%%%%%%%%%%%%%%%%%%%%%%%%%%%%%%%%%%%%%%%%%%%%
\index{Wave propagation!neutrons|(}%
\index{Neutrons!wave propagation|(}%

\index{Schrodinger@Schrödinger equation!microscopic}%
The scalar wavefunction $\psi(\r,t)$
\nomenclature[2t020]{$t$}{Time}%
\nomenclature[2r040]{$\r$}{Position}%
\nomenclature[1ψ030 2r040 2t02]{$\psi(\r,t)$}{Microscopic neutron wavefunction}%
of a free neutron
is governed by the microscopic Schrödinger equation
\begin{equation}\label{ESchrodi}
  i\hbar\partial_t \psi(\r,t)
  = \left\{-\frac{\hbar^2}{2m}\Nabla^2+V(\r)\right\} \psi(\r,t).
\end{equation}
By assuming a time-independent potential $V(\r)$,
we have excluded inelastic scattering.
Therefore we only need to consider monochromatic waves
with given frequency~$\omega$.
\nomenclature[1ω 020]{$\omega$}{Frequency of incident radiation}%
In consequence, we have a stationary wavefunction
\begin{equation}\label{Estationarywave}
  \psi(\r,t) = \psi(\r)\e^{-i\omega t}.
\end{equation}
\nomenclature[1ψ030 2r040 0]{$\psi(\r)$}{Stationary wavefunction}%
The minus sign in the exponent of the phase factor
is an inevitable consequence of the standard form of the Schrödinger equation,
and is therefore called the \E{quantum-mechanical sign convention}.
\index{Wave propagation|seealso {Sign convention}}%
\index{Sign convention!wave propagation}%
\index{Conventions|see {Sign convention}}%
For electromagnetic radiation
usage is less uniform.
While most optics textbooks
have adopted the quantum-mechanical convention~\cref{Estationarywave},
in X-ray crystallography
the conjugate phase factor $\e^{+i\omega t}$ is prefered.
This \E{crystallographic sign convention} has also been chosen
in influential texts on GISAXS (e.g.\ \cite{ReLL09}).
Here, however, we are concerned not only with X-rays,
but also with neutrons,
and therefore we need to leave the Schrödinger equation~\cref{ESchrodi} intact.
Thence:

\Note
{\indent In this manual, and in the program code of \BornAgain,
the quantum-mechanical sign convention~\cref{Estationarywave} is chosen.
This has implications for the sign of the imaginary part of the
refractive index,
\index{Index of refraction|see {Refractive index}}
\index{Refractive index!sign convention}%
as explained in \Cref{Sabsorption}.}

Inserting \cref{Estationarywave} in \cref{ESchrodi},
we obtain the stationary Schrödinger equation
\begin{equation}\label{EstatSchrodi}
  \left\{-\frac{\hbar^2}{2m}\Nabla^2+V(\r)-\hbar\omega\right\} \psi(\r) = 0.
\end{equation}
\nomenclature[2m020]{$m$}{Neutron mass}%
\nomenclature[2v130 2r040]{$V(\r)$}{Microscopic optical potential}%
\index{Potential|see {Optical potential}}%
\index{Optical potential!nuclear (microscopic)}%
The \E{nuclear} (or \E{microscopic})
\E{optical potential} $V(\r)$,
in a somewhat ``naive conception'' \cite[p.~7]{Sea89},
consists of a sum of delta functions,
representing Fermi's ``pseudopotential''.
\index{Fermi's pseudopotential}%
The superposition of the incident wave with the scattered waves
originating from each illuminated nucleus
results in \E{coherent forward scattering},
\index{Coherent forward scattering}%
in line with Huygens' principle.
\index{Huygens' principle}%

Coherent superposition also leads to \E{Bragg scattering}.
\index{Bragg scattering!by atomic lattices}%
However, Bragg scattering by atomic lattices only occurs at angles
far above the small-angle range covered in GISAS experiments.
Accordingly, it can be neglected in the analysis of GISAS data,
or at most, is taken into account as a loss channel.

Therefore,
we can neglect the atomic structure of $V(\r)$,
and perform some coarse graining to
arrive at a \E{continuum approximation}.
\index{Continuum approximation!neutron propagation}%
This is
similar to the passage from
the microscopic to the macroscopic Maxwell equations.
The details are intricate \cite{Sea89,Lax51},
but the result \cite[eq.~2.8.32]{Sea89} looks very simple:
The macroscopic field equation
has still the form of a stationary Schrödinger equation,
\index{Schrodinger@Schrödinger equation!macroscopic}%
\begin{equation}\label{EmacrSchrodi}
  \left\{-\frac{\hbar^2}{2m}\Nabla^2+v(\r)-\hbar\omega\right\} \psi(\r) = 0,
\end{equation}
\nomenclature[1ψ030 2r040 2t020]{$\psi(\r,t)$}{Coherent wavefunction}%
\nomenclature[2v020 2r040]{$v(\r)$}{Macrosopic optical potential}%
where $\psi$ now stands for the \E{coherent wavefunction}
\index{Coherent wavefunction}%
\index{Wave propagation!coherent}%
obtained by superposition of
incident and forward scattered states,
and $v(\r)$ is the \E{macroscopic optical potential}.
\index{Optical potential!macroscopic}%
This potential is weak, and slowly varying compared to atomic length scales.
It can be rewritten in a number of ways,
especially in terms of a
\E{bound scattering length density}
\index{Scattering length density}%
$\rho_s(\r)$ \cite[eq.\ 2.8.37]{Sea89},
\nomenclature[1ρ 034 2s000 2r040]{$\rho_s(\r)$}{Scattering length density}%
\begin{equation}
  v(\r)=\frac{2\pi \hbar^2}{m}\rho_s(\r),
\end{equation}
or of a \E{refractive index}~$n(\r)$
\nomenclature[2n020 2r040]{$n(\r)$}{Refractive index}%
\index{Refractive index} % !vs scattering length density}%
\index{Index of refraction|see {Refractive index}}%
defined by
\begin{equation}\label{EnRefrIndx}
  n(\r)^2\coloneqq 1-\frac{4\pi}{K^2}\rho_s(\r) = 1 -\frac{2m}{\hbar^2 K^2}v(\r).
\end{equation}
In the latter expression,
we introduced the \E{vacuum wavenumber}~$K$,
\nomenclature[2k120]{$K$}{Vacuum wavenumber, corresponding to the frequency~$\omega$}%
which is connected with the frequency~$\omega$ through the
\E{dispersion relation}
\begin{equation}
  \frac{\hbar^2 K^2}{2m} = \hbar\omega.
\end{equation}
Since we only consider stationary solutions~\cref{Estationarywave},
$\omega$ will not appear any further in our derivations.
Instead, we use~$K$ as the given parameter that characterizes the
incoming radiation.
In terms of $K$ and $n$,
the macroscopic Schrödinger equation \cref{EmacrSchrodi}
can be rewritten as
\Emph{
\begin{equation}\label{EnSchrodi}
  \left\{\Nabla^2+K^2n(\r)^2\right\}\psi(\r) = 0.
\end{equation}\vspace*{-10pt}
}
This equation is the starting point for the analysis of all
small-angle scattering experiments,
whether under grazing incidence (GISAS) or not (regular SAS).
\index{SAS|see {Small-angle scattering}}%
\index{Small-angle scattering}%

\index{Wave propagation!neutrons|)}%
\index{Neutrons!wave propagation|)}%

%%%%%%%%%%%%%%%%%%%%%%%%%%%%%%%%%%%%%%%%%%%%%%%%%%%%%%%%%%%%%%%%%%%%%%%%%%%%%%%%
\section{Neutron scattering in Born approximation}\label{SBornApprox}
%%%%%%%%%%%%%%%%%%%%%%%%%%%%%%%%%%%%%%%%%%%%%%%%%%%%%%%%%%%%%%%%%%%%%%%%%%%%%%%%

%===============================================================================
\subsection{The Born expansion}\label{SBornExpans}
%===============================================================================

\index{Born approximation|(}%

To describe an elastic scattering experiment,
we need to solve the Schrödinger equation~\cref{EnSchrodi}
under the asymptotic boundary condition
\begin{equation}\label{Escabouco}
  \psi(\r)
  \simeq \psi_\ti(\r) + f(\vartheta,\varphi)\frac{\e^{iKr}}{4\pi r}
  \text{~for~}r\to\infty,
\end{equation}
\nomenclature[1ψ034 2i000 2r040]{$\psi_\ti(\r)$}{Incident wavefunction}%
\nomenclature[2i000]{i}{Subscript ``incident''}%
where $\psi_\ti(\r)$ is the incident wave
as prepared by the experimental apparatus,
and the second term on the right-hand side is
the outgoing scattered wave
that carries information in form of the angular distribution
$f(\vartheta,\varphi)$.

For thermal or cold neutrons,
as for X-rays, the refractive index~$n$ is almost always
very close to~1.
This suggests a solution of the Schrödinger equation
by means of a perturbation expansion in powers of $n^2-1$.
This expansion is named after Max Born
who introduced it in quantum mechanics.\footnote
{It goes back to Lord Rayleigh
who devised it for sound,
and later also applied it to electromagnetic waves,
which resulted in his famous explanation of the blue sky.}

To carry out this idea, we rewrite the Schrödinger equation
once more so that it takes the form of a Helmholtz equation
\index{Helmholtz equation}%
with a perturbation term on the right side:
\begin{equation}\label{ESchrodiHelmholtz}
  \left(\Nabla^2+K^2\right)\psi(\r)
  = 4\pi\chi(\r)\psi(\r)
\end{equation}
with
\begin{equation}\label{EChiDef}
  \chi(\r) \coloneqq  \frac{K^2}{4\pi}\left(1-n^2(\r)\right).
\end{equation}
\nomenclature[1χ030 2r040]{$\chi(\r)$}{Perturbative potential,
for neutrons equal to the scattering-length density~$\rho_s$}%
This definition just compensates \cref{EnRefrIndx} so that $\chi=\rho_s$.
In the following, we prefer the notation~$\chi$
and the appellation \E{perturbative potential}
\index{Potential|see {Perturbation}}%
\index{Perturbation}%
over the scattering length density~$\rho_s$
to prepare for the generalization to the electromagnetic case.

Equation~\cref{ESchrodiHelmholtz} looks
like an inhomogeneous differential equation ---
provided we neglect for a moment that the unknown function~$\psi$
reappears on the right side.
The homogeneous equation
\begin{equation}\label{EHelmholtzHomog}
  \left(\Nabla^2+K^2\right)\psi(\r) = 0
\end{equation}
is solved by plane waves and superpositions thereof.
It applies in particular to the incident wave~$\psi_\ti$.

For an isolated inhomogeneity,
\begin{equation}\label{EHelmholtzForGreen}
  \left(\Nabla^2+K^2\right)G(\r,\r') = \delta(\r-\r')
\end{equation}
\nomenclature[2g130 2r040 2r041]{$G(\r,\r')$}{Green function}%
\index{Green function!homogeneous material}%
is solved by the Green function\footnote
{Verification under the condition $\r\ne0$
is a straightforward exercise in vector analysis.
For the special case $\r=0$,
one encloses the origin in a small sphere
and integrates by means of the Gauss-Ostrogadsky divergence theorem.
This explains the appearance of the factor $4\pi$.}
\begin{equation}\label{EGreens1}
  G(\r,\r') = \frac{\e^{iK|\r-\r'|}}{4\pi |\r-\r'|},
\end{equation}
which is an outgoing spherical wave centered at $\r'$.
Convoluting this function with the given inhomogeneity $4\pi\chi\psi$,
we obtain what is known as the Lippmann-Schwinger equation,
\index{Lippmann-Schwinger equation}%the formal solution
\begin{equation}\label{EPsiFormal}
  \psi(\r)
  = \psi_\ti(\r)
  + \int\!\d^3r'\, G(\r,\r') 4\pi\chi(\r')\psi(\r').
\end{equation}
This integral equation for $\psi(\r)$ improves
upon the original stationary Schrödinger equation \cref{ESchrodiHelmholtz}
in that it ensures the boundary condition~\cref{Escabouco}.
It can be resolved into an infinite series
by iteratively substituting the full right-hand side of~\cref{EPsiFormal}
into the integrand.
Successive terms in this series contain rising powers of $\chi$.
Since $\chi$ is assumed to be small, the series is likely to converge.
In \E{first-order Born approximation},
only the linear order in $\chi$ is retained,
\begin{equation}\label{EBorn}
  \psi(\r)
  \doteq \psi_\ti(\r)
  + 4\pi \int\!\d^3r'\, G(\r,\r') \chi(\r') \psi_\ti(\r').
\end{equation}
This is practically always adequate for
material investigations with X-rays or neutrons,
where the aim is to
deduce $\chi(\r')$ from the scattered intensity ${|\psi(\r)|}^2$.
Since detectors are always placed at positions $\r$
that are not illuminated by the incident beam,
we are only interested in the scattered wave field
\begin{equation}\label{EBornS}
  \psi_\text{s}(\r)
  \coloneqq
  4\pi \int\!\d^3r'\, G(\r,\r') \chi(\r') \psi_\ti(\r').
\end{equation}
\nomenclature[1ψ034 2s000 0 2r040]{$\psi_\text{s}(\r)$}{Scattered wavefunction}%
\nomenclature[2s000 0]{s}{Subscript ``scattered''}%

\index{Born approximation|)}%

%===============================================================================
\subsection{Far-field approximation}
%===============================================================================

\index{Far-field approximation|(}%

We can further simplify \cref{EBornS}
under the conditions of Fraunhofer diffraction:
\index{Fraunhofer approximation}%
the distance from the sample to the detector location~$\r$
must be much larger than the size of the sample.
Since the scattered wave $\psi_\text{s}(\r)$
only depends on $\r$ through the Green function~$G(\r,\r')$,
we shall derive a far-field approximation for the latter.

We choose the origin within the sample
so that the integral in~\cref{EBornS} runs over $\r'$ with $r'\ll r$.
This allows us to expand
\begin{equation}
  \left|\r-\r'\right|
  \doteq \sqrt{r^2-2\r\,\r'}
  \doteq r - \frac{\r\,\r'}{r}
  \equiv r - \frac{\k_\tf \r'}{K},
\end{equation}
\nomenclature[2f000]{f}{Subscript ``final'',
for outgoing waves scattered into the direction of the detector}%
\nomenclature[2k040]{$\k$}{wavevector}
where we have introduced the outgoing wavevector
\begin{equation}
  \k_\tf\coloneqq K\frac{\r}{r}.
\end{equation}
We apply this to~\cref{EGreens1},
\index{Green function!homogeneous material}%
and obtain in leading order the far-field Green function
\begin{equation}\label{EGreenFar}
  G_\text{far}(\r,\r')
  = \frac{\e^{iKr}}{4\pi r}\psi^*_\tf(\r')
\end{equation}
\nomenclature[2g134 2far]{$G_\text{far}(\r,\r')$}
  {Far-field approximation to the Green function $G(\r,\r')$}
where
\begin{equation}
  \psi_\tf(\r) \coloneqq  \e^{i\k_\tf \r}
\end{equation}
\nomenclature[1ψ034 2f000 2r040]{$\psi_\tf(\r)$}
  {Plane wave propagating from the sample towards the detector}%
is a plane wave propagating towards the detector,
and $\psi^*$ designates the complex conjugate of $\psi$.
With respect to $\r$, $G_\text{far}$ is an outgoing spherical wave.

The scattered wave~\cref{EBornS}
becomes in the far-field approximation
\begin{equation}\label{EsandwichC}
  \psi_\text{s,far}(\r)
  = \frac{\e^{iKr}}{r}
    \bra \psi_\tf|\chi|\psi_\ti\ket,
\end{equation}
\nomenclature[1ψ034 2s000 2far]{$\psi_\text{s,far}(\r)$}
  {Far-field approximation to the scattered wavefunction $\psi_\text{s}(\r)$}%
where we used Dirac notation for the transition matrix element
\index{Transition matrix}%
\begin{equation}\label{Etrama}
  \bra \psi_\tf|\chi|\psi_\ti\ket
  \coloneqq  \int\!\d^3r\, \psi^*_\tf(\r)\chi(\r)\psi_\ti(\r).
\end{equation}
\nomenclature[0$\langle$0]{{$\bra\ldots\vert\ldots\vert\ldots\ket$}}
  {Matrix element, defined as a volume integral}%
In order to reconcile conflicting sign conventions,
we will in the following rather use its complex conjugate
$\bra \psi_\ti|\chi|\psi_\tf\ket = \bra \psi_\tf|\chi|\psi_\ti\ket^*$.
Under the standard assumption
that the incident radiation is a plane wave
\begin{equation}\label{EPsi0Plane}
  \psi_\ti(\r)=\e^{i \k_\ti \r}
\end{equation}
with $k_\ti=K$,
the matrix element takes the form
\begin{equation}\label{Echiq}
  \bra \psi_\ti|\chi|\psi_\tf\ket
  = \int\!\d^3r\, {\rm e}^{-i\k_\ti\r}\chi(\r){\rm e}^{i\k_\tf\r}
  = \int\!\d^3r\, {\rm e}^{i\q\r}\chi(\r)
  \eqqcolon \chi(\q),
\end{equation}
\nomenclature[1χ030 2q040]{$\chi(\v{q})$}
  {Fourier transform of the perturbation potential $\chi(\r)$}%
where we have introduced the \E{scattering vector}\footnote
{With this choice of sign,
\index{Sign convention!scattering vector}%
$\hbar\q$ is the momentum
\index{Momentum transfer|see {Scattering vector}}%
\E{gained} by the scattered neutron,
and \E{lost} by the sample.
In much of the literature the opposite convention is prefered,
since it emphasizes the sample physics over the scattering experiment.
However, when working with two-dimensional detectors
it is highly desirable to express pixel coordinates
\index{Coordinate system}
and scattering vector components
with respect to equally oriented coordinate axes,
which can only be achieved by the convention~\cref{Eq}.}
\index{Scattering vector}%
\begin{equation}\label{Eq}
  \q\coloneqq \k_\tf-\k_\ti
\end{equation}
\nomenclature[2q040]{$\q$}{Scattering vector}%
and the notation $\chi(\q)$ for
the Fourier transform of the perturbative potential,
\index{Optical potential!Fourier transform}%
which is what small-angle neutron scattering basically measures.

\index{Far-field approximation|)}%


%===============================================================================
\subsection{Differential cross section}\label{SdiffCross}
%===============================================================================

In connection with \cref{EBorn} we mentioned
that a scattering experiment measures intensities~${|\psi(\r)|}^2$.
We shall now restate this in a more rigorous way.
In the case of neutron scattering,
one actually measures a \E{probability flux}.
We define it in arbitrary relative units as
\begin{equation}\label{EdefJ}
  \v{J}(\r) \coloneqq  \psi^*\frac{\Nabla}{2i}\psi - \psi\frac{\Nabla}{2i}\psi^*.
\end{equation}
\nomenclature[2j150 2r040]{$\v{J}(\r)$}{Probability flux}
\index{Flux!incident and scattered}%
The ratio of the scattered flux hitting an infinitesimal detector area
$r^2\d\Omega$ to the incident flux is expressed as a
\E{differential cross section}
\index{Cross section}%
\begin{equation}\label{Exsectiondef}
  \frac{\d\sigma}{\d\Omega}
  \coloneqq  \frac{r^2 J(\r)}{J_\ti}.
\end{equation}
\nomenclature[1ω120]{$\Omega$}{Solid angle}%
\nomenclature[1σ020]{$\sigma$}{Scattering or absorption cross section}%
% TODO RESTORE XREF
% The geometric factors that are needed to
% convert $\d\sigma/\d\Omega$ into detector counts will be discussed
% below in \cref{SdetImg}.

With \cref{EPsi0Plane}, the incident flux is
\begin{equation}\label{EJi}
  \v{J}_\ti = \k_\ti.
\end{equation}
With \cref{EsandwichC}, the scattered flux at the detector is
\begin{equation}\label{EJr}
  \v{J}(\r)
  = \v{\hat r}\frac{K}{r^2}
    {\left|\bra\psi_\ti|\chi|\psi_\tf\ket\right|}^2.
\end{equation}
From \cref{Exsectiondef} we obtain
the generic differential cross section
of elastic scattering in first order Born approximation,
\Emph{
\begin{equation}\label{Exsection}
  \frac{\d\sigma}{\d\Omega}
  =  {\left|\bra\psi_\ti|\chi|\psi_\tf\ket\right|}^2.
\end{equation}\vspace*{-5pt}
}
As we shall see below,
it holds not only for plane waves governed
by the vacuum Helmholtz equation~\cref{EHelmholtzHomog},
but also for distorted waves.

In the plane-wave case \cref{Echiq} considered here,
the differential cross section is just the squared modulus
of the Fourier transform of the perturbative potential,
\begin{equation}\label{Ecross1}
  \frac{\d\sigma}{\d\Omega}
  = {\left| \chi(\q) \right|}^2.
\end{equation}

%%%%%%%%%%%%%%%%%%%%%%%%%%%%%%%%%%%%%%%%%%%%%%%%%%%%%%%%%%%%%%%%%%%%%%%%%%%%%%%%
\section{Detector images [TO REVISE]}\label{SdetImg}
%%%%%%%%%%%%%%%%%%%%%%%%%%%%%%%%%%%%%%%%%%%%%%%%%%%%%%%%%%%%%%%%%%%%%%%%%%%%%%%%

\def\tc{\text{c}}

%-------------------------------------------------------------------------------
\begin{figure}[t]
\begin{center}
\includegraphics[width=.5\textwidth]{fig/drawing/experimental_geometry.png}
\end{center}
\caption{Experimental geometry with a two-dimensional pixel detector.}
\label{FexpGeom}
\end{figure}
%-------------------------------------------------------------------------------

To conclude this chapter on the foundations of small-angle scattering,
we shall derive the geometric factors
that allow us to convert differential cross sections into detector counts.
We shall also discuss how to present data on a physically meaningful scale.

%===============================================================================
\subsection{Pixel coordinates, scattering angles, and $\symbf{q}$ components}
%===============================================================================

We assume that scattered radiation is detected in a flat,
two-dimensional detector
that generates histograms on a rectangular grid,
consisting of $n\cdot m$ pixels of constant width and height,
as sketched in \cref{FexpGeom}.
This figure also shows the coordinate system
\index{Conventions|see {Coordinate system}}%
\index{Coordinate system}%
according to unanimous GISAS convention,
with $z$ normal to the sample plane,
and with the incident beam in the $xz$ plane.
The origin is at the center of the sample surface.
We suppose that the detector is mounted perpendicular to the $x$ axis
at a distance $L$ from the sample position.
%the $x$ axis intersects the detector plane at $(L,y_\tc,z_\tc)$.
The real-space coordinate at the center of pixel $(i,j)$ is $(L,y_i,z_i)$.
Each pixel has a width~$\Delta y$ and a height~$\Delta z$.
\index{Detector!pixel coordinate}%
\index{Pixel|see {Detector}}%

The affine-linear conversion from pixel indices $i,j$ to pixel coordinates $x_i,y_i$
requires an experimental calibration of the origin.
\Work{How is this done in \BornAgain?}
\index{Detector!calibration}%

Since the differential scattering cross section \cref{Exsectiondef}
is given with respect to a solid-angle element $\d\Omega$,
we need to express the scattered wavevector $\k_\tf$ in spherical coordinates,
using the horizontal azimuth angle~$\phi_\tf$
and the vertical glancing angle $\alpha_\tf$.
The projection of $(\alpha_\tf,\phi_\tf)$ into
the detector plane~$(y,z)$ is known as the \E{gnomonic projection}.
\index{Gnomonic projection}%
\index{Projection!wave vector to pixel coordinate}%
\index{Mapping!wave vector to pixel coordinate}%
\index{Transformation!wave vector to pixel coordinate}%
From elementary trigonometry one finds
\begin{equation}\label{Eyzdet}
  \begin{array}{lcl}
  y &=& L \tan\phi_\tf,\\
  z &=& (L/ \cos\phi_\tf) \tan\alpha_\tf.
  \end{array}
\end{equation}
\cref{Fconstalphi} shows lines of equal $\alpha_\tf$,~$\phi_\tf$
in the detector plane.
To emphasize the curvature of the constant-$\alpha_\tf$ lines,
scattering angles up to more than 25$^\circ$ are shown.
In typical SAS or GISAS,
scattering angles are much smaller,
and therefore the mapping between pixel coordinates and
scattering angles is in a good first approximation linear.
Of course \BornAgain\ is not restricted to this linear regime,
but uses the exact nonlinear mapping~\cref{Eyzdet}.

%-------------------------------------------------------------------------------
\begin{figure}[t]
\begin{center}
\includegraphics[width=.47\textwidth]{fig/drawing/SAS_const_alphi.ps}
\end{center}
\caption{Lines of constant $\alpha_\tf$ (red) or $\phi_\tf$ (blue)
in the detector plane,
for a planar detector at distance~$L$ from the sample.
The black dot indicates the beamstop location for
the central incident beam (SAS geometry, $\hat\k_\ti = \hat x$).}
\label{Fconstalphi}
\end{figure}
%-------------------------------------------------------------------------------

To determine the scattering vector $\q_{ij}$
that corresponds to a pixel $(i,j)$,
we need to express the outgoing wavevector~$\k_\tf$ as function of $y$ and~$z$.
This can be done either by inverting \cref{Eyzdet}
and inserting the so obtained $\alpha_\tf(y,z)$ and $\phi_\tf(y)$ in
\begin{equation}\label{Ekf_by_angle}
  \k_\tf=K\left(\begin{array}{c}
   \cos\alpha_\tf\cos\phi_\tf\\
   \cos\alpha_\tf\sin\phi_\tf\\
   \sin\alpha_\tf\end{array}\right),
\end{equation}
or much more directly by using geometric similarity in Cartesian coordinates.
The result is rather simple:
\Emph{
\begin{equation}\label{Ekf_by_pixel}
  \k_\tf=\frac{K}{\sqrt{L^2+y^2+z^2}}\left(\begin{array}{c}
   L\\
   y\\
   z\end{array}\right).
\end{equation}
\vspace*{-5pt}}


%-------------------------------------------------------------------------------
\begin{figure}[t]
\begin{center}
\includegraphics[width=.47\textwidth]{fig/drawing/SAS_const_q_x.ps}
\hfill
\includegraphics[width=.47\textwidth]{fig/drawing/SAS_const_q_yz.ps}
\end{center}
\caption{Lines of constant $q_x$ (left), $q_y$ or $q_z$ (right),
in units of the incident wavenumber $K=2\pi/\lambda$,
for a planar detector.
SAS geometry as in Fig.~\protect\ref{Fconstalphi}.}
\label{Fconstq}
\end{figure}
%-------------------------------------------------------------------------------

The transform \cref{Eqalgo} between pixel coordinates $y$,~$z$
and physical scattering vector components $q_y$, $q_z$
is nonlinear, due to the square-root term in the denominator of~\cref{Ekf_by_pixel}.
For $y,z\ll L$, however, nonlinear terms loose importance.

%-------------------------------------------------------------------------------
\begin{figure}[t]
\begin{center}
\includegraphics[width=1\textwidth]{fig/ff2/ff_det_box.pdf}
\end{center}
\caption{Simulated detector image for small-angle scattering from
uncorrelated cuboids (right rectangular prisms).
  \index{Box (form factor)}
  \index{Cuboid (form factor)}
  \index{Prism (form factor)!reactangular (Box)}
  \index{FormFactorBox@\Code{FormFactorBox}}
The incoming wavelength is 0.1~nm.
The prisms have edge lengths $L_y=L_z=10$~nm;
the length $L_x$, in beam direction, is varied as shown above the plots.
\index{Circular modulation!of detector image as function of $q_x$}
The circular modulation comes from a factor $\sinc(q_x L_x/2)$
in the cuboid form factor, with $q_x$ given by~\cref{Eqxasy}.}
\label{Fdetbox}
\end{figure}
%-------------------------------------------------------------------------------

The left detector frame in \cref{Fconstq}
shows circles of constant values of $\pm q_x$.
For given steps in $q_x$, the distance between adjacent circles
increases towards the detector center.
From \cref{Eq} and \cref{Ekf_by_pixel},
one finds asymptotically for $y,z\to L$
that $q_x$ goes with the square of the two other components of the scattering vector,
\begin{equation}\label{Eqxasy}
  \frac{q_x}{K}
  \doteq \frac{y^2+z^2}{2 L^2}
  \doteq \frac{q_y^2 + q_z^2}{2K^2}.
\end{equation}
Therefore, under typical small angle conditions $y,z\to L$
the dependence of the scattering signal on $q_x$ is unimportant:
one basically measures $\chi(\q)\simeq \chi(0,q_y,q_z)$.
The exception, for sample structures with long correlations in $x$~direction,
is illustrated in~\cref{Fdetbox}.

%-------------------------------------------------------------------------------
\begin{figure}[t]
\begin{center}
\includegraphics[width=.47\textwidth]{fig/drawing/SAS_const_p_yz.ps}
\end{center}
\caption{The outer contour of the blue and red grid
shows the border of a square detector image
after transformation into the physical coordinates $q_y$,~$q_z$.
The blue and red curves correspond to horizontal and vertical lines in the detector.}
\label{Fconstp}
\end{figure}
%-------------------------------------------------------------------------------

As anticipated in \cref{Eqxasy},
the other two components of $\q$ are in first order linear in the pixel coordinates,
\begin{equation}
  \frac{q_y}{K}=\frac{y}{L}\left(1-\frac{y^2+z^2}{2L^2}+\ldots\right),
\end{equation}
and similarly for~$q_z$.
The nonlinear correction terms lead to the pincushion distortion
shown in the right detector frame in \cref{Fconstq}.
\index{Distortion!of $q_x$, $q_y$ grid in detector plane}%
\index{Detector!distortion of $q_x$, $q_y$ grid}%
\index{Pincushion distortion}%

Since pixel coordinates are meaningful only
with respect to a specific experimental setup,
users may wish to transform detector images
towards the physical coordinates $q_y$ and~$q_z$.
As shown in \cref{Fconstp},
this would yield a barrel-shaped illuminated area
in the $q_y$,~$q_z$~plane.

To summarize this section,
the wavevector $\q_{ij}$ can be determined from the pixel indices
through the following steps:
\begin{equation}\label{Eqalgo}
  \begin{array}{cl}
      (i,j)&\\
      \downarrow&\mbox{calibrate of origin, then employ affine-linear mapping}\\
      (y,z)\\
      \downarrow&\mbox{use (\protect\ref{Ekf_by_pixel})}\\
      \k_\tf&\\
      \downarrow&\mbox{use (\protect\ref{Eq})}\\
      \q&\\
  \end{array}
\end{equation}

\Note{\indent Transforming detector images
  from pixel coordinates into the $q_y$,~$q_z$~plane is not implemented in \BornAgain,
  and not on our agenda.
  We would, however, like to hear about use cases.}

\Emph{\indent When simulating and fitting experimental data with \BornAgain,
detector images remain unchanged.
All work is done in terms of reduced pixel coordinates $y/L$ and~$z/L$.
Corrections are applied to the simulated, not to the measured data.}

\Work{\indent \ldots show how to plot $q$ grid on top of detector image \ldots}

%===============================================================================
\subsection{Intensity transformation}
%===============================================================================

The solid angle under which a detector pixel
is illuminated from the sample is in linear approximation
\begin{equation}
  \Delta\Omega
  = \cos\alpha_\tf\:\Delta\alpha_\tf\,\Delta\phi_\tf
  = \cos\alpha_\tf
    \left|\frac{\partial(\alpha_\tf,\phi_\tf)}{\partial(y,z)}\right|
    \Delta y \,\Delta z
  = \cos^3\!\alpha_\tf\, \cos^3\!\phi_\tf\: \frac{\Delta y \,\Delta z}{L^2}.
\end{equation}
\index{Detector!illumination angle correction factor}%
\index{Illumination!detector}%
Altogether,
the expected count rate in detector pixel $(i,j)$ is proportional to
\begin{equation}\label{EItrafo_cos}
  I_{ij} = \cos^3\!\alpha_\tf\, \cos^3\!\phi_\tf\:
          \frac{\partial\sigma}{\partial\Omega}(\q_{ij}),
\end{equation}
where we have omitted constant factors $L^{-2}$, $\Delta y$ and $\Delta z$.
Using pixel coordinates instead of angles, this can be rewritten as
\Emph{%
\begin{equation}\label{EItrafo_pix}
  I_{ij} = \left( 1+\frac{y^2+z^2}{L^2}\right)^{-3/2}
          \frac{\partial\sigma}{\partial\Omega}\bigl(\q_{ij}(y,z)\bigr).
\end{equation}
\vspace*{-2pt}}
%(usually as the centre between the transmitted
%and the specularly reflected beam spot).

%Tradition wants that raw data be \E{treated} or \E{reduced}
%before they are \E{analyzed}.
%In our case, raw data reduction would comprise
%the transform \cref{Eqalgo} of pixel coordinates into scattering vectors,
%and the accompanying renormalization \cref{EItrafo} of pixel counts.

%===============================================================================
\subsection{Symmetry}
%===============================================================================

\Work{to write ... and to continue in other chapters}

\index{Small-angle scattering|)}%
%\fi
